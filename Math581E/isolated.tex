\documentclass[12pt, reqno]{amsart}

\usepackage[utf8]{inputenc}
\usepackage{colorprofiles}
\usepackage[a-2b,mathxmp]{pdfx}[2018/12/22]
\hypersetup{pdfstartview=}

% Color comments!
\usepackage{xcolor}
\usepackage{fullpage}

\usepackage{hyperref}
\hypersetup{colorlinks=true, linkcolor=blue, citecolor=blue}
\usepackage[alphabetic,backrefs,lite]{amsrefs}

\usepackage{amssymb, amsmath, amsthm, amsfonts, mathrsfs, mathtools}

\usepackage{amscd}   % for commutative diagrams
\usepackage{tikz-cd} % for complicated commutative diagrams
\usepackage[all,cmtip]{xy}

\usepackage[shortlabels]{enumitem}


% Note: if you have problems with the Sha in cyrillic font,
% change the following ``paragraph'' to \newcommand{\Sha}{{\operatorname{Sha}}}
\DeclareFontEncoding{OT2}{}{} % to enable usage of cyrillic fonts
\newcommand{\textcyr}[1]{%
 {\fontencoding{OT2}\fontfamily{wncyr}\fontseries{m}\fontshape{n}\selectfont #1}}
\newcommand{\Sha}{{\mbox{\textcyr{Sh}}}}
% note: cmr might work in place of wncyr


\newcommand\commentr[1]{{\color{red} \sf [#1]}}
\newcommand\commentb[1]{{\color{blue} \sf [#1]}}
\newcommand\commentm[1]{{\color{magenta} \sf [#1]}}

\newcommand{\bvedit}[1]{{\color{magenta}   #1}}
\newcommand{\bianca}[1]{{\color{magenta} \sf $\clubsuit\clubsuit\clubsuit$ Bianca: [#1]}}
\newcommand{\margBi}[1]{\normalsize{{\color{red}\footnote{{\color{magenta}#1}}}{\marginpar[{\color{red}\hfill\tiny\thefootnote$\rightarrow$}]{{\color{red}$\leftarrow$\tiny\thefootnote}}}}}
\newcommand{\Bianca}[1]{\margBi{(Bianca) #1}}


% Theorems

\newtheorem{lemma}{Lemma}[section]
\newtheorem{theorem}[lemma]{Theorem}
\newtheorem{propo}[lemma]{Proposition}
\newtheorem{prop}[lemma]{Proposition}
\newtheorem{cor}[lemma]{Corollary}
\newtheorem{conj}[lemma]{Conjecture}
\newtheorem{claim}[lemma]{Claim}
\newtheorem{claim*}{Claim}
\newtheorem{rmk}[lemma]{Remark}
\newtheorem{remark}[lemma]{Remark}
\newtheorem{thm}[lemma]{Theorem}
\newtheorem{defn}[lemma]{Definition}
\newtheorem{example}[lemma]{Example}

%Characters
\newcommand{\A}{{\mathbb A}}
\newcommand{\G}{{\mathbb G}}
\newcommand{\bbH}{{\mathbb H}}
\newcommand{\PP}{{\mathbb P}}
\newcommand{\C}{{\mathbb C}}
\newcommand{\F}{{\mathbb F}}
\newcommand{\Q}{{\mathbb Q}}
\newcommand{\R}{{\mathbb R}}
\newcommand{\Z}{{\mathbb Z}}

\newcommand{\Xbar}{{\overline{X}}}
\newcommand{\Qbar}{{\overline{\Q}}}
\newcommand{\Zhat}{{\hat{\Z}}}
\newcommand{\Zbar}{{\overline{\Z}}}
\newcommand{\kbar}{{\overline{k}}}
\newcommand{\Kbar}{{\overline{K}}}
\newcommand{\ksep}{{k^{\operatorname{sep}}}}
\newcommand{\Fbar}{{\overline{\F}}}

\newcommand{\kk}{{\mathbf k}}

\newcommand{\pp}{{\mathfrak p}}
\newcommand{\qq}{{\mathfrak q}}
\newcommand{\mm}{{\mathfrak m}}

% mathcal characters
\newcommand{\calA}{{\mathcal A}}
\newcommand{\calB}{{\mathcal B}}
\newcommand{\calC}{{\mathcal C}}
\newcommand{\calD}{{\mathcal D}}
\newcommand{\calE}{{\mathcal E}}
\newcommand{\calF}{{\mathcal F}}
\newcommand{\calG}{{\mathcal G}}
\newcommand{\calH}{{\mathcal H}}
\newcommand{\calI}{{\mathcal I}}
\newcommand{\calJ}{{\mathcal J}}
\newcommand{\calK}{{\mathcal K}}
\newcommand{\calL}{{\mathcal L}}
\newcommand{\calM}{{\mathcal M}}
\newcommand{\calN}{{\mathcal N}}
\newcommand{\calO}{{\mathcal O}}
\newcommand{\calP}{{\mathcal P}}
\newcommand{\calQ}{{\mathcal Q}}
\newcommand{\calR}{{\mathcal R}}
\newcommand{\calS}{{\mathcal S}}
\newcommand{\calT}{{\mathcal T}}
\newcommand{\calU}{{\mathcal U}}
\newcommand{\calV}{{\mathcal V}}
\newcommand{\calW}{{\mathcal W}}
\newcommand{\calX}{{\mathcal X}}
\newcommand{\calY}{{\mathcal Y}}
\newcommand{\calZ}{{\mathcal Z}}
\newcommand{\OO}{{\mathcal O}}


% Math operators
\DeclareMathOperator{\tr}{tr}
\DeclareMathOperator{\Tr}{Tr}
\DeclareMathOperator{\trdeg}{tr deg}
\DeclareMathOperator{\lcm}{lcm}
\DeclareMathOperator{\supp}{supp}
\DeclareMathOperator{\Frob}{Frob}
\DeclareMathOperator{\coker}{coker}
\DeclareMathOperator{\rk}{rk}
\DeclareMathOperator{\Char}{char}
\DeclareMathOperator{\inv}{inv}
\DeclareMathOperator{\nil}{nil}
\DeclareMathOperator{\im}{im}
\DeclareMathOperator{\re}{Re}
\DeclareMathOperator{\End}{End}
\DeclareMathOperator{\END}{\bf End}
\DeclareMathOperator{\Lie}{Lie}
\DeclareMathOperator{\Hom}{Hom}
\DeclareMathOperator{\Ext}{Ext}
\DeclareMathOperator{\HOM}{\bf Hom}
\DeclareMathOperator{\Aut}{Aut}
\DeclareMathOperator{\Gal}{Gal}
\DeclareMathOperator{\Ind}{Ind}
\DeclareMathOperator{\Cor}{Cor}
\DeclareMathOperator{\Res}{Res}
\DeclareMathOperator{\Norm}{Norm}
\DeclareMathOperator{\Br}{Br}
\DeclareMathOperator{\Gr}{Gr}
\DeclareMathOperator{\cd}{cd}
\DeclareMathOperator{\scd}{scd}
\DeclareMathOperator{\Sel}{Sel}
\DeclareMathOperator{\Cl}{Cl}
\DeclareMathOperator{\divv}{div}
\DeclareMathOperator{\ord}{ord}
\DeclareMathOperator{\Sym}{Sym}
\DeclareMathOperator{\Div}{Div}
\DeclareMathOperator{\Pic}{Pic}
\DeclareMathOperator{\Jac}{Jac}
\DeclareMathOperator{\Num}{Num}
\DeclareMathOperator{\Spec}{Spec}
\DeclareMathOperator{\Proj}{Proj}
\DeclareMathOperator{\Hilb}{Hilb}

% Commands

\numberwithin{equation}{section}
\numberwithin{table}{section}

\newcommand{\defi}[1]{\textsf{#1}} % for defined terms

\title{Low Genus Curves with many Isolated Points}

\author{John Leo}
\address{Halfaya Research, Bellevue, WA 98006, USA}
\email{leo@halfaya.org}
\urladdr{https://www.halfaya.org}

% \keywords{key1, key2, key3}
\subjclass[2020]{11G30}

\begin{document}

\begin{abstract}
Bourdon et al.\ \cite{Bourdon2019} in 2019 introduced the concept of
\textit{isolated points} on varieties, a generalization of rational
points, and proved that any ``nice'' curve over a number field has
finitely many isolated points. We examine the state of the art related
to low genus curves with a large number of isolated points.
\end{abstract}

\maketitle

\section{Introduction}

Bourdon et al.\ \cite{Bourdon2019} in 2019 introduced the concept of
\textit{isolated points} on varieties, a generalization of rational
points, and proved that any ``nice'' curve over a number field has
finitely many isolated points. Natural questions to ask are how to find
specific curves (starting with lower genera) with large numbers of
such points, and how to count the points and/or determine them
explicity. This report surveys work that has been done for the special
case of rational points, as well as more recent work on the general
case. In its current state the survey mostly references the papers and
does not attempt to summarize the techniques used.

\section{Background}

We provide here some fundamental definitions and facts. For details
see \cite{Viray2024}.

Let $k$ be a number field and $\kbar$ be its
algebraic closure. Let $C/k$ denote a \textit{nice} (smooth,
projective, and geometrically integral over $k$) curve (variety of
dimension $1$ over $k$). A \textit{closed point} on $C$ is the Galois
orbit of a geometric point in $X(\kbar)$, and its \textit{degree} is
the cardinality of this orbit. A rational point is the special case in
which the degree is $1$.

The number of rational points on a curve is governed by its genus:
\begin{enumerate}
  \item \textbf{Genus $0$.} There are either zero or infinitely many
    rational points.
  \item \textbf{Genus $1$.} The set of ratonal points is a finitely
    generated abelian group (the Mordell-Weil Theorem). Furthermore the possible
    torsion groups are completely understood for a curve over $\Q$
    (\cites{Mazur1977, Mazur1978}) and are known to be bounded in
    general (\cite{Merel1996}).
  \item \textbf{Genus $\ge 2$.} There are finitely many rational points
    (Faltings's Theorem (\cite{Faltings1983})).
\end{enumerate}

When $d > 1$, the number of degree $d$ points can be infinite even
when the genus is $\ge 2$. However all but finitely many of these
points can be paramerized in one of two ways.
\begin{enumerate}
  \item \textbf{$\PP^1$-parameterized points}. A closed point $x \in
    C$ is $\PP^1$-parameterized if there exists a morphism $\pi : \C \to \PP^1$ with
    $\deg(\pi) = \deg(x)$ and $\pi(x) \in \PP^1(k)$.

  \item \textbf{AV-parameterized points}. A degree $d$ closed point $x
    \in C$ is AV-parameterized if there exists a positive rank abelian
    subvariety $A \subset \Pic^0_C$ such that $[x] + A \subset W^d =
    \im(\Sym^d_C \to \Pic^d_C)$.
\end{enumerate}

A point that is neither $\PP^1$- nor AV-parameterized is called
\textit{isolated}, and \cite{Bourdon2019}*{Theorem~4.2} prove that
there are only finitely many isolated points (of any degree) on a
given curve, and furthermore there are infinitely many points of
degree $d$ iff there exists a degree $d$ $\PP^1$-parameterized point.
These are referenced as Corollary~4.3 and Theorem~4.4 of
\cite{Viray2024}.

Furthermore a closed point $x \in C$ is called \textit{sporadic} if
there are only finitely many closed points of $C$ with degree at most
$\deg(x)$. By \cite{Bourdon2019}*{Theorem~4.2}, every sporadic point
must also be isolated; however, the converse need not hold.

Given that any curve has only finitely many isolated points, a natural
question to ask is how to find specific curves with large numbers of
such points. Since the concept of isolated points is a relatively new
one, introduced in \cite{Bourdon2019}, we first consider the rational
(degree $1$) special case, which has been studied for a longer period
of time.


\section{Curves with Many Rational Points}

As noted above, finiteness of the number of rational points is controlled by the genus.

\subsection{Genus \texorpdfstring{$< 2$}{< 2}}

A curve of genus 0 either has zero or infinitely many rational points,
so this case is uninteresting.

For the genus 1 case, by the Mordell-Weil Theorem the set of rational
points is a finitely-generated abelian group and thus finite iff the
rank of the group is zero. The number of rational points is then equal
to the size of the torsion group. In the case $k=\Q$, Mazur's theorem
states that the largest such group is $\Z/2\Z \times \Z/8\Z$ of order
$16$. There are currently three such curves in LMFDB (\cite{lmfdb}),
which can be found by searching for ``Elliptic Curves over $\Q$'' with
the given rank and torsion.

For arbitrary nunber fields, the situation is more
complicated. \cite{Parent1999} proves that the size of the torsion
group is bounded by an exponential in the degree $d$ of the field
extension. The bound is conjectured to be polynomial in $d$; see
\cite{Clark2018}.

For quadratic extensions, the combined work of \cites{Kamienny1992a,
  Kamienny1992b, Kenku1988} shows that the torsion group has size at
most 24 and in this case must be $\Z/2\Z \times \Z/12\Z$. For cubic
extensions, \cite{Derickx2021} proves that the largest possible
torsion group is $\Z/2\Z \times \Z/14\Z$ of size 28. Less is known
about higher degress. See the introduction to \cite{Genao2022} for an
overview.

As far as specific examples go, searching LMFDB for ``Elliptic curves over
$\Q(\alpha)$'' of rank $0$ and sorting by torsion, one finds the largest
group to have order 37, followed by 36, 32, and three curves with
28. The degree of the minimal polynomial for these examples ranges
from 3 to 6. There are two curves in the database over quadradic
fields with torsion order 24, one over $\Q(\sqrt{6})$ and the other
over $\Q(\sqrt{-15})$. There is one cubic curve in the datebase over a
cubic field with torsion order 28.

\subsection{Genus \texorpdfstring{$\ge 2$}{≥ 2}}

By Faltings's Theorem there are only finitely many rational points. For
genus 2 the curve with the most such points in LMFDB has only 39. However Stoll
(\cite{Muller2016}) exhibits a curve with at least 642 rational
points, which is apparently still the record. The paper describes how
to find such points for a given curve, but does not say how to find
curves with likely many points. The referenced announcement
\cite{Stoll2008} merely states ``in the course of a systematic search
for curves with many rational points in several families constructed
by Noam Elkies, I discovered the following curve of genus 2''. More
details can be found in \cite{Stoll2015}.

For higher genera, \cite{Caporaso1995} is a good, if dated, reference.

\section{Isolated and Sporadic Points}

As noted above, isolated and sporadic points are relatively new
concepts. Work so far has concentrated on the modular curves $X_0(N)$
and $X_1(N)$. These have the advantage that their points correspond to
isomorphism classes of elliptic curves, and thus concretely to
$j$-invariants. CM and non-CM cases are distinguished as they
require different techniques.

Bourdon and collaborators focus on $X_1(N)$, starting with
\cite{Bourdon2019}. Section~8 of this paper focuses on classifying
non-cuspidal non-CM isolated points with rational $j$-invariant. The
conditions on the $j$-invariant of curves corresponding to such
isolated points can then be used to search LMFDB for appropriate
elliptic curves, which then correspond to isolated points.

\cite{Bourdon2021} focus on conditions for sporadic points and
considers both CM and non-CM cases.

\cite{Bourdon2024} focus on isolated points of odd degree, and
proves (Theorem~2) that such points can have only two specific (for
the non-CM case) or three specific (for the CM case)
$j$-invariants. Only for the two non-CM cases have isolated points
been verified to exist.

Finally \cite{Bourdon2025} develop an algorithm to determine whether a
non-CM rational $j$-invariant is isolated. Running this on LMFDB they
determine, given restrictions on the conductor, that at most four such
rational $j$-invariants exist (Theorem~2); they are also known to
exist. The authors conjecture that these are the only possible
$j$-invariants (removing the conductor restrictions); the case for odd
degree was already proven in \cite{Bourdon2024}, and \cite{Ejder2022}
handles the case $X_1(\ell^n)$ with $\ell$ prime.

\cite{Box2023} focus on $X_0(N)$ with several specific values of $N$,
and explicit calculation of isolated points. No more than 4 isolated
points are demonstrated for any example curve.  \cite{Khawaja2024}
provide criteria for determining isolated points of more general
curves, but do not attempt to find curves with many isolated
points.

As of this writing (November 2024), the LMFDB references isolated
points only in the beta version \cite{lmfdb_beta} in a new section
``Modular Curves''; there is a text box ``Isolated'' in the section
``Modular curve low-degree point search results''. However isolated
points are not yet computed, so no information can be retrieved.

\bibliography{isolated.bib}

\end{document}

