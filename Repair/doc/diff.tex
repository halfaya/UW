\documentclass{article}

\usepackage{amssymb}
\usepackage{agda}

\begin{document}

\title{Differences of Proofs}
\author{John Leo}
\maketitle

\section{Introduction}

This document is a place to record my thoughts on the Coq proof
repair project headed by Talia Ringer, Nate Yazdani, and Dan
Grossman.  Everything in it is preliminary.

\section{Difference of Terms}

Consider the following scenario.  We are given type $A$ and a fixed
term $a : A$.  For a given type $B$ (for the time being, not depending
on $a$) we would like to construct a function (called a ``patch'') $f
: A \to B$. Now say that you are given $b : B$. Clearly one can set
$f(x) = b$ for all $x$.  However what we would like to do is find a
``minimal'' body of $f$ where we are trying to minimize some measure
such as the size of its AST, and if we could replace any occurrence of
$a$ in $b$ with $x$ then that would reduce the size by $|a| - |x| =
|a| - 1$.  We can thus try to find the maximal number of occurrences
of $a$ in $b$ that we can abstract and replace with the
variable $x$.

\subsection{Examples}

For now my examples will be written in Agda, as the original examples
in the user study use Ltac whereas the actual work is done in Gallina
so I'd need to generate the Gallina myself; also Agda is easier to
read, at least for me.  But the Gallina code is very similar and can
be generated later if it matters.

The first example from the user study is the following.

\begin{verbatim}
≤transWeak : {n m p : ℕ} → n ≤ m → m ≤ p → n ≤ p + 1
≤transWeak {p = p} a b = ≤-trans (≤-trans a b) (m≤m+n p 1)

≤trans : {n m p : ℕ} → n ≤ m → m ≤ p → n ≤ p
≤trans a b = ≤-trans a b

≤transWeakPatch : ({n m p : ℕ} → n ≤ m → m ≤ p → n ≤ p) →
                  ({n m p : ℕ} → n ≤ m → m ≤ p → n ≤ p + 1)
≤transWeakPatch P {n} {m} {p} b c = ≤-trans (P b c) (m≤m+n p 1)

≤transWeak' : {n m p : ℕ} → n ≤ m → m ≤ p → n ≤ p + 1
≤transWeak' = ≤transWeakPatch ≤trans
\end{verbatim}

In this case our original term $a$ is ≤-trans.

\subsection{Strengthened Conclusions}

All eight of the examples in the CSE 503 project, as well as most
later examples (such as oldMinimal/newMinimal in email) are of the
same form, which could be characterized as strengthening or in general
modifying the conclusion of a theorem given a set of hypthoses.  The
types involved are pi types, and we are given terms $a : (x : X) \to
A$ and $b : (x : X) \to B$ where $A$ and $B$ can depend on $x$.  Note
that $X$ is identical for both $a$ and $b$.  We view it here as a
single sigma type, but in the original form of the it is curried so
that $a$ and $b$ have pure pi types.

\section{Alternate Datatype Formulations}


\end{document}
